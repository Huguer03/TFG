% Options for packages loaded elsewhere
% Options for packages loaded elsewhere
\PassOptionsToPackage{unicode}{hyperref}
\PassOptionsToPackage{hyphens}{url}
\PassOptionsToPackage{dvipsnames,svgnames,x11names}{xcolor}
%
\documentclass[
  basque,
  letterpaper,
  DIV=11,
  numbers=noendperiod]{scrartcl}
\usepackage{xcolor}
\usepackage{amsmath,amssymb}
\setcounter{secnumdepth}{5}
\usepackage{iftex}
\ifPDFTeX
  \usepackage[T1]{fontenc}
  \usepackage[utf8]{inputenc}
  \usepackage{textcomp} % provide euro and other symbols
\else % if luatex or xetex
  \usepackage{unicode-math} % this also loads fontspec
  \defaultfontfeatures{Scale=MatchLowercase}
  \defaultfontfeatures[\rmfamily]{Ligatures=TeX,Scale=1}
\fi
\usepackage{lmodern}
\ifPDFTeX\else
  % xetex/luatex font selection
\fi
% Use upquote if available, for straight quotes in verbatim environments
\IfFileExists{upquote.sty}{\usepackage{upquote}}{}
\IfFileExists{microtype.sty}{% use microtype if available
  \usepackage[]{microtype}
  \UseMicrotypeSet[protrusion]{basicmath} % disable protrusion for tt fonts
}{}
\makeatletter
\@ifundefined{KOMAClassName}{% if non-KOMA class
  \IfFileExists{parskip.sty}{%
    \usepackage{parskip}
  }{% else
    \setlength{\parindent}{0pt}
    \setlength{\parskip}{6pt plus 2pt minus 1pt}}
}{% if KOMA class
  \KOMAoptions{parskip=half}}
\makeatother
% Make \paragraph and \subparagraph free-standing
\makeatletter
\ifx\paragraph\undefined\else
  \let\oldparagraph\paragraph
  \renewcommand{\paragraph}{
    \@ifstar
      \xxxParagraphStar
      \xxxParagraphNoStar
  }
  \newcommand{\xxxParagraphStar}[1]{\oldparagraph*{#1}\mbox{}}
  \newcommand{\xxxParagraphNoStar}[1]{\oldparagraph{#1}\mbox{}}
\fi
\ifx\subparagraph\undefined\else
  \let\oldsubparagraph\subparagraph
  \renewcommand{\subparagraph}{
    \@ifstar
      \xxxSubParagraphStar
      \xxxSubParagraphNoStar
  }
  \newcommand{\xxxSubParagraphStar}[1]{\oldsubparagraph*{#1}\mbox{}}
  \newcommand{\xxxSubParagraphNoStar}[1]{\oldsubparagraph{#1}\mbox{}}
\fi
\makeatother


\usepackage{longtable,booktabs,array}
\usepackage{calc} % for calculating minipage widths
% Correct order of tables after \paragraph or \subparagraph
\usepackage{etoolbox}
\makeatletter
\patchcmd\longtable{\par}{\if@noskipsec\mbox{}\fi\par}{}{}
\makeatother
% Allow footnotes in longtable head/foot
\IfFileExists{footnotehyper.sty}{\usepackage{footnotehyper}}{\usepackage{footnote}}
\makesavenoteenv{longtable}
\usepackage{graphicx}
\makeatletter
\newsavebox\pandoc@box
\newcommand*\pandocbounded[1]{% scales image to fit in text height/width
  \sbox\pandoc@box{#1}%
  \Gscale@div\@tempa{\textheight}{\dimexpr\ht\pandoc@box+\dp\pandoc@box\relax}%
  \Gscale@div\@tempb{\linewidth}{\wd\pandoc@box}%
  \ifdim\@tempb\p@<\@tempa\p@\let\@tempa\@tempb\fi% select the smaller of both
  \ifdim\@tempa\p@<\p@\scalebox{\@tempa}{\usebox\pandoc@box}%
  \else\usebox{\pandoc@box}%
  \fi%
}
% Set default figure placement to htbp
\def\fps@figure{htbp}
\makeatother


% definitions for citeproc citations
\NewDocumentCommand\citeproctext{}{}
\NewDocumentCommand\citeproc{mm}{%
  \begingroup\def\citeproctext{#2}\cite{#1}\endgroup}
\makeatletter
 % allow citations to break across lines
 \let\@cite@ofmt\@firstofone
 % avoid brackets around text for \cite:
 \def\@biblabel#1{}
 \def\@cite#1#2{{#1\if@tempswa , #2\fi}}
\makeatother
\newlength{\cslhangindent}
\setlength{\cslhangindent}{1.5em}
\newlength{\csllabelwidth}
\setlength{\csllabelwidth}{3em}
\newenvironment{CSLReferences}[2] % #1 hanging-indent, #2 entry-spacing
 {\begin{list}{}{%
  \setlength{\itemindent}{0pt}
  \setlength{\leftmargin}{0pt}
  \setlength{\parsep}{0pt}
  % turn on hanging indent if param 1 is 1
  \ifodd #1
   \setlength{\leftmargin}{\cslhangindent}
   \setlength{\itemindent}{-1\cslhangindent}
  \fi
  % set entry spacing
  \setlength{\itemsep}{#2\baselineskip}}}
 {\end{list}}
\usepackage{calc}
\newcommand{\CSLBlock}[1]{\hfill\break\parbox[t]{\linewidth}{\strut\ignorespaces#1\strut}}
\newcommand{\CSLLeftMargin}[1]{\parbox[t]{\csllabelwidth}{\strut#1\strut}}
\newcommand{\CSLRightInline}[1]{\parbox[t]{\linewidth - \csllabelwidth}{\strut#1\strut}}
\newcommand{\CSLIndent}[1]{\hspace{\cslhangindent}#1}

\ifLuaTeX
\usepackage[bidi=basic]{babel}
\else
\usepackage[bidi=default]{babel}
\fi
% get rid of language-specific shorthands (see #6817):
\let\LanguageShortHands\languageshorthands
\def\languageshorthands#1{}


\setlength{\emergencystretch}{3em} % prevent overfull lines

\providecommand{\tightlist}{%
  \setlength{\itemsep}{0pt}\setlength{\parskip}{0pt}}



 


\KOMAoption{captions}{tableheading}
\makeatletter
\@ifpackageloaded{caption}{}{\usepackage{caption}}
\AtBeginDocument{%
\ifdefined\contentsname
  \renewcommand*\contentsname{Aurkibidea}
\else
  \newcommand\contentsname{Aurkibidea}
\fi
\ifdefined\listfigurename
  \renewcommand*\listfigurename{Irudien zerrenda}
\else
  \newcommand\listfigurename{Irudien zerrenda}
\fi
\ifdefined\listtablename
  \renewcommand*\listtablename{Taulen zerrenda}
\else
  \newcommand\listtablename{Taulen zerrenda}
\fi
\ifdefined\figurename
  \renewcommand*\figurename{Irudia}
\else
  \newcommand\figurename{Irudia}
\fi
\ifdefined\tablename
  \renewcommand*\tablename{Taula}
\else
  \newcommand\tablename{Taula}
\fi
}
\@ifpackageloaded{float}{}{\usepackage{float}}
\floatstyle{ruled}
\@ifundefined{c@chapter}{\newfloat{codelisting}{h}{lop}}{\newfloat{codelisting}{h}{lop}[chapter]}
\floatname{codelisting}{Zerrendatua}
\newcommand*\listoflistings{\listof{codelisting}{Zerrendatuen zerrenda}}
\makeatother
\makeatletter
\makeatother
\makeatletter
\@ifpackageloaded{caption}{}{\usepackage{caption}}
\@ifpackageloaded{subcaption}{}{\usepackage{subcaption}}
\makeatother
\usepackage{bookmark}
\IfFileExists{xurl.sty}{\usepackage{xurl}}{} % add URL line breaks if available
\urlstyle{same}
\hypersetup{
  pdftitle={BEC simulazioa},
  pdfauthor={Hugo González González},
  pdflang={eu},
  pdfkeywords={TFG, Grade of Physics, Physics Lab, BEC},
  colorlinks=true,
  linkcolor={blue},
  filecolor={Maroon},
  citecolor={Blue},
  urlcolor={Blue},
  pdfcreator={LaTeX via pandoc}}


\title{BEC simulazioa}
\author{Hugo González González}
\date{2026-02-18}
\begin{document}
\maketitle
\begin{abstract}
This analysis provides \ldots{} \textbar{}
\end{abstract}


\section{Oinarrizko teoria}\label{oinarrizko-teoria}

Atal hau (Pitaevskii eta Stringari 2016) liburuan oinarrituta dago.
Aztertuko ditugu oinarrizko gauzak GPE ekuazioa ulertu ahal izateko.

\subsection{Ez-uniforme Bose gasak
T=0K}\label{ez-uniforme-bose-gasak-t0k}

Ikertuko ditugu Bose gas diluituen dinamika.

\subsubsection{Gross-Pitaevskii ekuazioa
(GPE)}\label{gross-pitaevskii-ekuazioa-gpe}

Elektromagnetismoan egiten dugun bezala, \(\hat{\Psi}(\mathbf{r}, t)\)
eremu eragilea eremu klasiko batekin, \(\Psi_0(\mathbf{r}, t)\),
ordezkatuko dugu. Azken honi kondentsatuaren uhin-funtzioa edo
ordena-parametroa deitzen zaio. Orain behar dugu ekuazio bat duena
funtzio klasikoaren dinamika. Hau lortzeko gogoratu Heisemberg-en
irudian hurrengo erlazioa bete behar duela \(\hat{\Psi}\):

\[
i\hbar \frac{\partial}{\partial t} \hat{\Psi}(\mathbf{r}, t) = [\hat{\Psi}(\mathbf{r}, t), \hat{H}]
\]

\begin{equation}\phantomsection\label{eq-p1}{
= \left[ -\frac{\hbar^2 \nabla^2}{2m} + V_{ext}(\mathbf{r}, t) + \int \hat{\Psi}^\dagger(\mathbf{r}', t) V(\mathbf{r}' - \mathbf{r}) \hat{\Psi}(\mathbf{r}', t) d\mathbf{r}' \right] \hat{\Psi}(\mathbf{r}, t),
}\end{equation}

non \(V_{ext}(\mathbf{r}, t)\) kanpo-potentziala den eta
\(V(\mathbf{r}' - \mathbf{r})\)

Aldaketa egin ahal izateko, Bornen hurbilketa bete beharko da.
Honetarako \(V_{eff}\) bete beharko ditu energia txikiko sakabanaketaren
baldintzak eta \(\Psi_0\) era motelean aldatuko da
elkarrekintza-potentzialaren barrutian. Hau betetzen bada
\(\mathbf{r'}\) \(\mathbf{r}\)-rekin aldatu dezakegu.

\begin{equation}\phantomsection\label{eq-GPE}{
i\hbar \frac{\partial}{\partial t} \Psi(\mathbf{r}, t) = \left( -\frac{\hbar^2 \nabla^2}{2m} + V_{ext}(\mathbf{r}, t) + g|\Psi(\mathbf{r}, t)|^2 \right) \Psi(\mathbf{r}, t)
}\end{equation}

non ordena-parametroa \(g = \int V_{eff} \; d\mathbf{r}\) den. Hau
s-wave sakabanaketaren \(a\) parametroaren funtzioan jar dezakegu:

\begin{equation}\phantomsection\label{eq-p3}{
g = \frac{4 \pi \hbar^2 a}{m}
}\end{equation}

non \(a\) s-uhinen sakabanatze-luzera den eta \(m\) atomo baten masa.

Hau izango da Maxwell ekuazioen analogoa baina energia eta momentuaren
erlazioa fotoiena izan beharrean De Brogliren erlazioa beteko dute.

Zenbait baldintza bete beharko dira Ekuazioa~\ref{eq-GPE} erabili ahal
izateko.

\begin{itemize}
\tightlist
\item
  BEC-a sortzeko partikula kopuru handia behar ditugu.
\item
  Temperatura txikia izatea behar dugu partikulen artean sortzen diren
  elkarrekintzak txikiak izateko.
\item
  Gure uhin funtzioa dago normalizatuta hurrengo moduan:
  \(\int |\Psi|^2 \; d\mathbf{r} = N\)
\item
  Ekuazioa bete egingo da \(\mathbf{r} \gg a\) denean.
\end{itemize}

Hau guztia betetzen bada, gasaren dentsitatea kondentzatuarenarekin bat
etorriko da:

\begin{equation}\phantomsection\label{eq-density}{
n(\mathbf{r}) = |\Psi(\mathbf{r})|^2
}\end{equation}

Beste era bat izango da egoera estazionario bat inposatzen

\begin{equation}\phantomsection\label{eq-p5}{
\delta \left[ -i\hbar \int \Psi^* \frac{\partial}{\partial t} \Psi \, d\mathbf{r}dt + \int E \, dt \right] = 0
}\end{equation}

akzioari. Honekin izango dugun ekuazioa

\begin{equation}\phantomsection\label{eq-p6}{
i\hbar \frac{\partial \Psi(\mathbf{r}, t)}{\partial t} = \frac{\delta E}{\delta \Psi^*(\mathbf{r}, t)}
}\end{equation}

izango da, non energia funtzionalaren ordena-parametroa

\begin{equation}\phantomsection\label{eq-p7}{
E = \int \left( \frac{\hbar^2}{2m}|\nabla\Psi|^2 + V_{ext}(\mathbf{r})|\Psi|^2 + \frac{g}{2}|\Psi|^4 \right) d\mathbf{r}
}\end{equation}

Orain ikusiko ditugu Ekuazioa~\ref{eq-GPE}-ren kontserbazio legeak.
Lehenengo, kontuan izanda \(\int |\Psi|^2 \; d\mathbf{r} = N\) dela, hau
da, partikula kopurua konstante mantentzen dela. Beraz,
Ekuazioa~\ref{eq-GPE} \(\Psi^*\)-rekin biderkatuz probabilitatearen
kontserbazio legea lortzen da:

\begin{equation}\phantomsection\label{eq-kontinuidad}{
\frac{\partial n}{\partial t} + \text{div}\mathbf{j} = 0
}\end{equation}

non erabili dugun Ekuazioa~\ref{eq-density} eta sartu dugun korronte
dentsitatea:

\begin{equation}\phantomsection\label{eq-korrontea}{
\mathbf{j}(\mathbf{r}, t) = -\frac{i\hbar}{2m} (\Psi^* \nabla \Psi - \Psi \nabla \Psi^*) = n \frac{\hbar}{m} \nabla S
}\end{equation}

non \(S\) ordena-parametroaren fasea den. Gogoratu, ordena-parametroa
funtzio konplexua izanda honela idatzi dezakegula:

\begin{equation}\phantomsection\label{eq-uf}{
\Psi(\mathbf{r},t) = \sqrt{n(\mathbf{r},t)}e^{i S(\mathbf{r},t)}.
}\end{equation}

Ekuazio honetatik lortu ahal dugu zein den kondentsatuaren abiadura
fasearen bitartez:

\begin{equation}\phantomsection\label{eq-abiadura}{
\mathbf{v_s}(\mathbf{r},t) = \frac{\hbar}{m} \nabla S.
}\end{equation}

Gainera, ikusi dezakegunez hau irrotazionala izango da, zentsua daukana,
azkenean superfluido guztiek propietate hau erakusten baitute.

Jakinda energia ere kontserbatu egingo dela lortu ahal izango dugu
momentu dentsitatearen ekuazioa.

\begin{equation}\phantomsection\label{eq-momentu-kontserbazioa}{
m \frac{\partial j_i}{\partial t} + \frac{\partial \Pi_{ik}}{\partial x_k} = -n \frac{\partial V_{ext}}{\partial x_i},
}\end{equation}

non

\begin{equation}\phantomsection\label{eq-momentu-tentsorea}{
\Pi_{ik} = \frac{\hbar^2}{4m^2} \left[ \frac{\partial \Psi}{\partial x_i} \frac{\partial \Psi^*}{\partial x_k} - \Psi \frac{\partial^2 \Psi^*}{\partial x_i \partial x_k} + \text{c.c.} \right] + \frac{gn^2}{2} \delta_{ik},
}\end{equation}

momentuaren fluxu tentsorea da. Ohartu, kanpo indar barik momentua
kontserbatu egingo dela.

Erabilgarria da fasearentzako ekuazio bat lortzea. Hau lortzen da
Ekuazioa~\ref{eq-uf} Ekuazioa~\ref{eq-GPE} sartuz:

\begin{equation}\phantomsection\label{eq-S}{
\hbar \frac{\partial}{\partial t} S + \left( \frac{1}{2} m \mathbf{v}_s^2 + V_{ext} + gn - \frac{\hbar^2}{2m\sqrt{n}} \nabla^2 \sqrt{n} \right) = 0.
}\end{equation}

Konturatu Ekuazioa~\ref{eq-density} eta Ekuazioa~\ref{eq-S} osatzen
dutela ekuazio multzo bat analogoa dena GPE ekuazioarekin. \(\hbar\)
ekuazioan sartzen da dentsitatearen gradientearen bitartez, honi deitzen
zaio ``presio kuantikoa''.

Ekuazioa~\ref{eq-GPE}-ren soluzio geldikorra forma simple bat du

\begin{equation}\phantomsection\label{eq-guf}{
\Psi(\mathbf{r},t) = \Psi_0(\mathbf{r}) e^{-\frac{i\mu t}{\hbar}}.
}\end{equation}

Denboraren menpekotasuna potentzial kimikoa zehazten du

\begin{equation}\phantomsection\label{eq-muux28Eux29}{
\mu = \frac{\partial E}{\partial N},
}\end{equation}

guzti honekin Gross-Pitaevskii ekuazioa honela idatzi dezakegu

\begin{equation}\phantomsection\label{eq-egonGPE}{
\left( -\frac{\hbar^2 \nabla^2}{2m} + V_{ext}(\mathbf{r}) + g|\Psi_0(\mathbf{r})|^2 - \mu \right) \Psi_0(\mathbf{r}) = 0,
}\end{equation}

non suposatu egin dugu gure potentziala ez dela denboraren menpekoa.
\(\mu\)-ren balioa normalizazio baldintzak ezarriko digu.

\subsection{Thomas-Fermi limitea}\label{thomas-fermi-limitea}

Ekuazioa~\ref{eq-S} ekuazioan ikusi dugu parte bat presio kuantiko
modukoa izango dela. Honek bi zatitan banatu dezakegu: \(gn\) interakzio
partean eta Bohm-en potentzialean (\(Q\)). Honen forma hurrengoa da:

\begin{equation}\phantomsection\label{eq-Q}{
Q = - \frac{\hbar^2}{2m} \frac{\nabla^2 \sqrt{n}}{\sqrt{n}}.
}\end{equation}

Thomas-Fermi limitean, \(\nabla^2 \sqrt{n}/\sqrt{n}\) magnitudea
berreskuratze luzera \(\xi = \hbar / \sqrt{2mgn}\) (hau erakusten du
noiz izango dugun kondentsatua a la ez, oso txikia, edo orden berekoa,
bada sistemaren luzera karakteristikoa interakzioak eta gero ez du
denborarik berreskuratzeko eta apurtzen da kondentsatua) baino askoz
handiagoa bada (hau da, \(\sqrt{n}\) oso era motelean aldatzen bada
denbora eta espazioan zehar) \(Q\) guztiz arbuiagarria izango da eta
berridatz dezakegu Ekuazioa~\ref{eq-S} abiadura eremu bat bezala:

\begin{equation}\phantomsection\label{eq-TF}{
m \frac{\partial \mathbf{v}_s}{\partial t} + \nabla \left( \frac{1}{2}m\mathbf{v}_s^2 + V_{ext} + gn \right) = 0 
}\end{equation}

non ekuazio hau bat datorren Eulerren ekuazioarekin, fluido batentzat
kanpo potentzial baten eraginean eta biskositate barik. Fluido honen
presioa \(P = gn^2/2\) eta soinuaren abiadura \(c=\sqrt{gn/m}\).
Horretaz gain, berridatzi dezakegu Ekuazioa~\ref{eq-kontinuidad}
\(\mathbf{v}_s\)-ren menpean

\begin{equation}\phantomsection\label{eq-kontinuidad_vs_rho}{
\frac{\partial n}{\partial t} + \text{div}(v_sn) = 0,
}\end{equation}

ekuazio bikote hauek hurrengo atalean ikusiko dugunez superfluido baten
hidrodinamikari datorkie.

Thomas-Fermi hurbilketan oinarrizko egoerak forma erreza hartzen du,
\(v_s = 0\). Hau da, energia zinetikoa guztiz arbuiatuz lortuko dugu.
Era honetan izango dugu:

\begin{equation}\phantomsection\label{eq-5.23}{
gn(\mathbf{r}) + V_{ext}(\mathbf{r}) = \mu
}\end{equation}

non \(\mu\) oinarrizko egoerari dagokion potentzial kimikoa den. Kanpo
eremurik ez daukagunean ailegatzen gara Bogoliubov erlazioari
\(\mu = gn\).

\subsubsection{Bortex interakzio txikiko Bose
gasan}\label{bortex-interakzio-txikiko-bose-gasan}

Esan dugunez, GPE deskribatuta dago Ekuazioa~\ref{eq-guf}-ren bitartez,
hau da, magnitude orden bat denborarekiko menpekotasuna duena, non
menpekotasuna fase global bat deskribatzen duen potentziak kimikoaren
funtzioa dena.

Orain aztertuko ditugu bortez soluzioak. Hauek ematen dira gure
kondentsatuaren dentsitatea zerorantz doanean. Hauek ematen diren
eremuak, berreskuratze luzera baino txikiagoko eskualdetan gertatzen
dira eta kontuan izan behar dugu presio kuantikoaren terminoa
GPE-n.~Normalean, azkar biratzen duen erreferentzia sisteman izango dira
egonkorrak, sistema horretan energiaren funtzionala zero baita.

Superfluidoetan dauden biraketak ez dira solido zurrunean dauden
bezalakoak \(\mathbf{v} = \mathbf{\Omega} \times \mathbf{r}\), baizik
eta deskribatuta egongo da bortizitatea difusioarekin:
\(\text{curl}\mathbf{v} = 2\mathbf{\Omega}\). Hau konturatzen bagara,
kontraesan bat izango da superfluidoak irrotazionalak direla esan
genuelako, hau da, espero dugu hauek ez biratzea solido zurruna bezala.

Demagun dugula \(R\) erradioko eta \(L\) luzerako zilindro batean gasa
konfinatuta. Izango dugun soluzioa hurrengo formakoa izango da:

\begin{equation}\phantomsection\label{eq-zuf}{
\Psi_0(\mathbf{r}) = e^{is\varphi}|\Psi_0(r)|,
}\end{equation}

non sartu ditugun koordenatu zilindrikoak \(z\), \(r\) eta \(\varphi\).
\(s\) zenbaki osoa da, gure uhin-funtzioa balio bakarra izan dezan.
Konturatzen bagara Ekuazioa~\ref{eq-zuf} \(l_z\)-ren autobalioa izango
da \(l_z = s \hbar\) izaten. Beraz, bortex osoaren momentu angeluarra
\(L_z = N s \hbar\) izango da. Ekuazioa~\ref{eq-zuf} adierazten du gas
bat biratzen hurrengo abiadura tangentzialarekin

\begin{equation}\phantomsection\label{eq-abiadura-tangen}{
v_s = \frac{\hbar}{m}\frac{s}{r}.
}\end{equation}

Konturatzen bagara hau guztiz desberdina da solido zurrunean
geneukanarekin, hau \(1/r\)-ren proportzionala baita
(Irudia~\ref{fig-vel-eremuak}).

\begin{figure}

\centering{

\pandocbounded{\includegraphics[keepaspectratio]{Index_files/figure-latex/fig-vel-eremuak-output-1.pdf}}

}

\caption{\label{fig-vel-eremuak}Abiadura-eremu tangentziala fluxu
irrotazionalerako (\(v_{\text{irr}}\)) eta errotazio rigidoko
(\(v_{\text{rig}}\)) fluxurako. Abiadura-eremu irrotazionala \(1/r\)
bezala dibergitzen da \(r \to 0\) denean. Hemen \(r\) eta \(v\) unitate
arbitrarioetan (a.u.) neurtzen dira.}

\end{figure}%

Ordezkatuz Ekuazioa~\ref{eq-zuf} Ekuazioa~\ref{eq-egonGPE}, lortuko
dugun adierazpena \(|\Psi_0|\)-rentzat:

\begin{equation}\phantomsection\label{eq-20}{
-\frac{\hbar^2}{2m} \frac{1}{r} \frac{d}{dr} \left( r \frac{d|\Psi_0|}{dr} \right) + \frac{\hbar^2 s^2}{2mr^2} |\Psi_0| + g|\Psi_0|^3 - \mu|\Psi_0| = 0.
}\end{equation}

Bortexaren distantzia handietara dakigu izan behar dugun dentsitatea
kondentsatuarena izango dela \(|\Psi_0| \to \sqrt{n}\). Hau dela eta,
sartuko dugu hurrengo funtzio adimensionala \(f(\eta)\) eta gure
dentsitatea:

\begin{equation}\phantomsection\label{eq-aduf}{
|\Psi_0| = \sqrt{n} f(\eta),
}\end{equation}

non \(\eta = r/\xi\) eta \(\xi = \hbar / \sqrt{2mgn}\) berreskuratze
luzera den. Funtzio honek betetzen duen ekuazioa

\begin{equation}\phantomsection\label{eq-f}{
\frac{1}{\eta} \frac{d}{d\eta} \left( \eta \frac{df}{d\eta} \right) + \left( 1 - \frac{s^2}{\eta^2} \right) f - f^3 = 0,
}\end{equation}

non muga baldintzak \(f(\eta\to\infty) = 1\) eta \(f(0) = 0\) izan behar
diren. Hau numerikoki ebasten badugu ikusten da
(Irudia~\ref{fig-bortex-profila}) nola \(r\) \(\xi\)-ren ordenakoa
denenean dentsitate aldaketa bortitza izango den.

\begin{figure}

\centering{

\pandocbounded{\includegraphics[keepaspectratio]{Index_files/figure-latex/fig-bortex-profila-output-1.pdf}}

}

\caption{\label{fig-bortex-profila}Ekuazioa~\ref{eq-f}-ren soluzio
numerikoak \(s=1\) eta \(s=2\) kasuetarako. Ikusten da nola
kondentsatuaren dentsitatea aldatzen den \(\eta\)-ren funtzioan.}

\end{figure}%

\subsection{Superfluidotasuna}\label{superfluidotasuna}

Aipatu dugunez, BEC-a superfluido baten moduan jokatzen du tenperatura
kritiko bat baino hotsago dagoenean. Parte honetan eztabaidatuko ditugu
Landauren superfluidoen hidrodinamika, eta nola teoria honek
ahalbidetzen dituen aurkitutako ordena-parametroaren fasearen
fluktuazioak BEC-aren teorian.

\subsubsection{Landauren irizpidea
superfluidotasunarentzako}\label{landauren-irizpidea-superfluidotasunarentzako}

Landauren superfluidoen teoria garatzeko, energiaren eta momentuaren
transformazio Galilearrak erabiliko erabiliko ditugu. \(E\) eta
\(\mathbf{P}\) izango dira energia eta momentua fluidoaren pausaguneko
sisteman \(K\). Orduan \(K'\) sisteman, \(\mathbf{v}\) abiaduraz
mugitzen dena \(K\) sistemarekiko, energia eta momentuaren adierazpenak:

\begin{equation}\phantomsection\label{eq-p8}{
E' = E - \mathbf{P.V} + \frac{1}{2} M V^2, \; \mathbf{P'} = \mathbf{P} - M \mathbf{v}
}\end{equation}

non \(M\) fluidoaren masa den.

Kontsidera dezagun fluido bat, kapilar baten zehar higitzen dena
\(\mathbf{v}\) abiadurarekin. Fluidoa biskositatea badu, galdutako
energia modelizatu ahal dugu higitzen den pertubazio baten moduan.
Perturbazio honen energia eta momentua fluidoaren pausaguneko sisteman
\(\mathbf{p}\) eta \(E_0 + \epsilon(\mathbf{p})\) izango dira, non
\(E_0\) pertubazioaren oinarrizko energia den eta
\(\epsilon(\mathbf{p})\) kitzikapenak sortutako energia. Kapilarraren
pausaguneko sisteman ikusiko dugu:

\begin{equation}\phantomsection\label{eq-p9}{
E' = E_0 + \epsilon(\mathbf{p}) + \mathbf{p.v} + \frac{1}{2} M v^2, \mathbf{P'} = \mathbf{p} + M \mathbf{v}
}\end{equation}

Ekuazioa~\ref{eq-p9} ikusi ahal dugu nola energia eta momentu aldaketak,
kapilarraren pausaguneko sisteman,
\(\epsilon(\mathbf{p}) + \mathbf{p.v}\) eta \(\mathbf{p}\) diren.
Orduan, \(\epsilon(\mathbf{p}) + \mathbf{p.v}\) negatiboa izan behar da
berezko kitzikapena izateko. Hau posiblea izateko,
\(v > \epsilon(\mathbf{p}) / p\) izan behar da. Suposatzen badugu
\(\mathbf{p}\) momentua kapilarrari transferitzen diola, hau bero moduan
disipatu egingo da eta fluidoa ez da egonkorra izango. Baina abiadura,

\begin{equation}\phantomsection\label{eq-vc}{
v_c = \text{min}_p \frac{\epsilon(\mathbf{p})}{p}
}\end{equation}

baino txikiago bada, kitzikapenak ezin dira agertu. Hau da Landauren
superfluidotasunaren baldintza.

\begin{equation}\phantomsection\label{eq-LSC}{
v < v_c
}\end{equation}

Fluidoaren eta kapilarraren abiadura erlatiboak, Ekuazioa~\ref{eq-vc}
balio kritikoa baino txikiagoa bada. Orduna marruskadura gabeko fluxua
izango dugu. Egoera honi, metaegonkorra deritzogu. Egonkorra da
bat-bateko kitzikapenekiko.

Interakzio ahuleko Bose gasa Landauren irispidea betetzen du, non
abiadura kritikoa soinuaren abiadura izango den. \(^4He\)-ren kasuan ere
beteko egingo du Landauren irispidea, baina kasu honetan abiadura
kritikoa soinuaren abiadura baino txikiagoa izango da, \(^4He\)
partikulen elkarrekintza handiagoa baita.

Demagun temperatura txikia dela. Gure fluidoaren propietateak izango
dira interakzio gabeko gas baten kitzikapenak (kuasipartikulak) oreka
termikoan bezalakoak. kuasipartikulak dira sistemaren masaren parte bat
garraiatzen dituztenak. Gainera, hauek talka egitean kapilarraren
hormekin energia disipatu egingo dute, fluido normal baten moduan.
Beraz, izango ditugu bi fase, bat normala \(\mathbf{v_n}\)-rekin eta
beste bat superfluidoa dena \(\mathbf{v_s}\), non abiadura hauek izango
dira kapilarraren pausaguneko sistemarekiko. Orduan kitzikapenen energia
\(\epsilon(\mathbf{p}) + \mathbf{p.}(\mathbf{v_s - v_n})\). Orekan
dauden kitzikapenen, \(\mathbf{p}\) banaketa funtzioa:

\begin{equation}\phantomsection\label{eq-Np}{
N_{\mathbf{p}} = \left[ exp\left(\frac{\epsilon(\mathbf{p}) + \mathbf{p.} (\mathbf{v_s - v_n})}{k_B T}\right) - 1 \right]^{-1}
}\end{equation}

Ekuazio hau betetzeko \(\mathbf{p}\)-ren balio guztietarako (hau da,
kitzikapenen energia positiboa izateko)
\(|\mathbf{v_s} - \mathbf{v_n}|\) abiadura kritikoa baino txikiagoa izan
beharko da. Orduan, \(N_\mathbf{p}\) beti izango da positiboa. Oreka
termodinamikoan gaudenean, superfluido fasearen eta fase normalaren
arteko marruskadurarik ez dago. Hau gertatzekotan, gero eta kitzikapen
gehiago sortuko lirateke eta, ondorioz, superfluido fasea apurtuko zen.

Kontuan izanda bi faseak batera daudela, fluidoaren masa dentsitatea
idatz dezakegu hurrengo moduan: \(\rho = \rho_n + \rho_s\) ((Pitaevskii
eta Stringari 2016) erabiltzen du \(\rho = n m\) dentsitatea hemendik
aurrera literatura orokorrean hau egiten baitenez, \(n\) erabili
beharrean). Likidoaren masa korrontea definitu dezakegu:

\begin{equation}\phantomsection\label{eq-mj}{
m \mathbf{j} = \rho_s \mathbf{v_s} + \rho_n \mathbf{v_n}
}\end{equation}

Hau erabili ahal dugu definitzeko fluido normalaren dentsitatea. Jar
gaitezen superfluidoaren pausaguneko sisteman, orduan
\(m\mathbf{j} = \rho_n \mathbf{v_n}\) izango dugu. Beraz, likidoak
dakarren momentu osoa
\(\mathbf{P} = \sum_{\mathbf{p}} N_{\mathbf{p}} \mathbf{p}\) izango da.
Onekin, Ekuazioa~\ref{eq-mj} berridatzi dezakegu:

\begin{equation}\phantomsection\label{eq-intmj}{
m \mathbf{j} = \int \; \frac{d \mathbf{p}}{(2 \pi \hbar)^3} \mathbf{p} N_{\mathbf{p}}
}\end{equation}

eta esan dugunez, superfluidoaren pausaguneko sisteman gaudenez

\begin{equation}\phantomsection\label{eq-intrhov}{
\rho_n \mathbf{v_n} = \int \; \frac{d \mathbf{p}}{(2 \pi \hbar)^3} \mathbf{p} N_{\mathbf{p}}
}\end{equation}

izango dugu. \(\mathbf{v_n}\) txikia izango denez (oreka termikoan
egoteko) \(N_{\mathbf{p}}\) seriean garatu dezakegu termino
linealerarte. Hau eginez, lortuko dugu fase normalaren masa dentsitatea

\begin{equation}\phantomsection\label{eq-masadentitatea}{
\rho_n = - \frac{1}{3} \int \; \frac{d \mathbf{p}}{(2 \pi \hbar)^3} \frac{d N_{\mathbf{p}}(\epsilon)}{d \epsilon} p^2,
}\end{equation}

isotropikoa izango dena \(\mathbf{p}\)-rekiko. Oso erabilgarria da,
kalkulatu ahal dugulako fase normalaren banaketa kitzikapenen funtzioen
bitartez. Baina zenbait limite dauzkagu, \(N_{\mathbf{p}}\) ez badago
ondo definitua, desorden handia dogoenean, beste era bat bilatu beharko
dugu kalkulatu ahal izateko fase normalaren dentsitatea.

\subsubsection{Bose-Einstein kondentsatua eta
superfluidotasuna}\label{bose-einstein-kondentsatua-eta-superfluidotasuna}

Orain nahi dugu BEC-aren eta superfluidoasunaren arteko erlazioa
aztertu. Lehen egin dugun moduan, kondentsatuaren \(\Psi_0\)
uhin-funtzioaren propietateak aztertu nahi ditugu, transformazio
Galilear bat egitean. Garrantzitsua da aipatzea, \(|\Psi_0|^2\)
dentsitatea uniformea izan arren, gure uhin-funtzioa ez da aldaezin bat
izango transformazio Galilear batekiko fase bat gehitu beharko
diogulako. Fase honen kalkulua egiteko erabili dezakegu
Ekuazioa~\ref{eq-p1}. Erreza da baieztatzea

\begin{equation}\phantomsection\label{eq-phiGalilear}{
\hat{\Psi}' = \hat{\Psi}(\mathbf{r-v}t,t) \exp \left[ \frac{i}{\hbar} \left( m \mathbf{v.r} - \frac{1}{2} m v^2 t \right) \right],
}\end{equation}

non \(\mathbf{v}\) bektore konstante bat den, soluzio bat dela. Hau
eremu eragilearen transformazio Galilearra da. ordena-parametroa eta
bere itxarotako balio izango dute transformazio berbera. Dakigunez, gure
fluidoa orekan dagoen erreferentzia sisteman, izango dugun uhin-funtzioa
\(\Psi_0 = \sqrt{n_0} e^{-i \mu t / \hbar}\), non \(n_0\)
\(\mathbf{r}\)-ren independiente den konstantea. Baina, fluidoa
\(\mathbf{v}\) abiadurarekin mugitzen den erreferentzia sisteman jarriz,
lehen esan dugunez modulua es da aldatuko, fasea ez bezala. gure
ordena-parametro berria \(\Psi_0 = \sqrt{n_0} e^{iS}\) non gure fase
berria

\begin{equation}\phantomsection\label{eq-6.15}{
S(\sqrt{r},t) = \frac{1}{\hbar} \left[ m \mathbf{v.r} - \left( \frac{1}{2} m v^2 + \mu \right) t \right]
}\end{equation}

izango den. Gainera, Hamilton-Jacobi ekuazio batetik hasi garenez
(Schrödinger-en ekuazioa), dakigu gure superfluidoaren abiadura fasearen
paraleloa izango dela:

\begin{equation}\phantomsection\label{eq-p6.16}{
\mathbf{v}_s = \frac{\hbar}{m} \nabla S.
}\end{equation}

Abiadura hau, era motelean aldatu ahal da denboran eta espazioan zehar.
Gainera, guztiz orokorra da, ez baititugu inolako hurbilketak egin.

\subsubsection{Superfluidoen hidrodinamikaren teoria
(T=0K)}\label{superfluidoen-hidrodinamikaren-teoria-t0k}

Temperatura nulua denean, portaera makroskopikoa erakusten duen ekuazioa
fluido irrotazional klasiko batena da. Kitzikapen termikoen gabezian,
deskribatu dezakegu gure sistema bi parametroen menpe, superfluidoaren
abiadurarekin eta bere masa dentsitatearekin. Masa dentsitatea lortuko
dugu jarraitutasun ekuazioaren bitartez,

\begin{equation}\phantomsection\label{eq-p6.17}{
\frac{\partial \rho}{t} + \text{div} (\mathbf{v}_s \rho) = 0
}\end{equation}

masaren kontserbazioa ezartzen duena. Abiadura eremuaren ekuazioa
lortzeko, fasea Ekuazioa~\ref{eq-6.15} betetzen duela erabiliko dugu

\begin{equation}\phantomsection\label{eq-p6.18}{
\hbar \frac{\partial S}{\partial t} = - \left( \frac{1}{2} m v^2 + \mu \right).
}\end{equation}

Aurreko atalean aurresan genuen bezala, ekuazio hau ez da bakarrik
baliagarria orekan (biak denbora eta posizioarekiko menpekotasunik ez
daukatenean), baizik eta hauek biak era motelean aldatzen direnean
denbora eta espazioan zehar. Orduan, aurreko adierazpenaren gradientea
hartuz eta Ekuazioa~\ref{eq-p6.16} adierazpena erabiliz lortuko dugu
abiadura eremuaren adierazpena

\begin{equation}\phantomsection\label{eq-p6.19}{
m \frac{\partial \mathbf{v}_s}{\partial t} + \nabla \left( \frac{1}{2} m \mathbf{v}_s^2 + \mu(\rho) \right) = 0
}\end{equation}

non potentziak kimikoa lokalki ebaluatzen da, kontuan izanda fluidoaren
dentsitate puntuala. Konturatzen bagara, adierazpen honek Eulerren
ekuazioaren antza dauka, baina kanpoko potentziala ez dagoenean. Beraz
\(V_{ext}\) bat gehitu ahal diogu

\begin{equation}\phantomsection\label{eq-p6.20}{
m \frac{\partial \mathbf{v}_s}{\partial t} + \nabla \left( \frac{1}{2} m \mathbf{v}_s^2 + \mu(\rho) + V_{ext} \right) = 0.
}\end{equation}

Oinarrizko egoera aztertzean, hau da \(\mathbf{v}_s = 0\) denean
berreskuaretzen dugu Thomas-Fermiren limitea non potentzial kimikoa

\begin{equation}\phantomsection\label{eq-p6.21}{
\mu(\rho(\mathbf{r})) + V_{ext}(\mathbf{r}) = \mu_0
}\end{equation}

izango den.

\subsubsection{hidrodinamika kuantikoa}\label{hidrodinamika-kuantikoa}

Ekuazioa~\ref{eq-p6.17} eta Ekuazioa~\ref{eq-p6.19} adierazpenak,
fluidoaren oszilazio txikiak soinu uhinak direla adierazten dute.
Mekanika kuantikoan, soinu uhin hauei fonoiak dira. Hau da, soinu uhinak
ohiko erregelen bitartez kuantizatzean lortzen ditugun kuasipartikulak
dira. Hidrodinamika klasikoaren ekuazioak kuantizatzeko, eremu klasikoak
kuantizatu beharko ditugu. Hau erabilgarria izango da interakzio handiko
Bose-Einstein kondentsatuak aztertzean.

Kuantizazio prozesua egiteko, Hamiltondar klasikoaren adierazpena
lortzea izango da erabilgarria. Likido baten energia

\begin{equation}\phantomsection\label{eq-p6.22}{
H = \int \; d\mathbf{r} \left( \frac{\rho}{2} (\nabla \phi)^2 + e(\rho) \right),
}\end{equation}

non \(\phi\) abiadura potentziala den \(\mathbf{v}_s = \nabla \phi\)
eran definitua, \(\rho\) fluidoaren dentsitatea eta potentzial kimikoa
sartu egin dugu fluidoaren bolumen unitateko energiaren bitartez
\(e(\rho)\); hauen arteko erlazioa \(\mu = m \; de/d\rho\) izanda.
Hamiltondarraren bariazioa eginez

\begin{equation}\phantomsection\label{eq-p6.23}{
\delta H = \int \left[ -\text{div}(\rho \mathbf{v}_s) \delta \phi + \left( \frac{1}{2} (\nabla \phi)^2 + \frac{\mu(\rho)}{m} \right) \right]
}\end{equation}

lortzen da. Honen bitartez Ekuazioa~\ref{eq-p6.17} eta
Ekuazioa~\ref{eq-p6.19} hurrengo eran jar daiteke:

\[
\frac{\partial \rho}{\partial t} = \frac{\delta H}{\delta \phi}, \; \frac{\partial \phi}{\partial t} = - \frac{\delta H}{\delta \rho},
\]

beraz argi dago \(\phi \text{ eta } \rho\) gure problemaren aldagai
konjokatuak direla. Landau 1941-tean, (Landau 1941) artikuluan, hurrengo
trukatzailea proposatu zuen

\begin{equation}\phantomsection\label{eq-p6.24}{
[\hat{\phi}(\mathbf{r}), \hat{\rho}(\mathbf{r'})] = -i \hbar \delta(\mathbf{r} - \mathbf{r'}).
}\end{equation}

Era honetan pasatu gara hidrodinamika klasikotik, hidrodinamika
kuantikora. Gainera fasea Ekuazioa~\ref{eq-p6.16} operadore batean
bihurtu egiten da ere:

\begin{equation}\phantomsection\label{eq-p6.25}{
\hat{S} = \frac{m}{\hbar} \hat{\phi}.
}\end{equation}

Hamiltondarra Ekuazioa~\ref{eq-p6.22}, ondo simetrizatuz gero hermitikoa
izan dadin, honela berridatzi dezakegu:

\begin{equation}\phantomsection\label{eq-p6.26}{
\hat{H} = \int \left( \nabla \hat{\phi} \frac{\hat{\rho}}{2} \nabla \hat{\phi} + e(\hat{\rho}) \right) d\mathbf{r}
}\end{equation}

Hidrodinamika kuantikoa arazoak ditu sistema makroskopikoaren uhin
bektore txikiak aztertzen ez ditugunean. Baina oso baliagarria da
Bogoliubov-en teoria ezin denean erabili, adibidez, interakzio handiko
BEC-tan gertatzen den moduan.

Helburua izango da, Ekuazioa~\ref{eq-p6.24} trukatzaile propietatea
erabiltzea \(\hat{\rho}\) eta \(\hat{\phi}\) sortze (\(\hat{b}\)) eta
deuseztatze (\(\hat{b}^\dagger\)) eragileen funtzioan adierazteko. Hau
lortzeko, erabilgarria da erabiltzea masa dentsitatearen karga
\(\hat{\rho}' = \hat{\rho} - \rho_0\) non nahi duguna da jakitea zenbat
aldatu den dentsitate lokala sistemaren dentsitate konstantearekiko.
Beraz karga dentsitatearen adierazpena

\begin{equation}\phantomsection\label{eq-p6.27}{
\hat{\rho}' = \frac{1}{\sqrt{2V}} \sum_{k \neq 0} A_k (\hat{b}_k e^{i\mathbf{k} \cdot \mathbf{r}} + \hat{b}_k^\dagger e^{-i\mathbf{k} \cdot \mathbf{r}})
}\end{equation}

izango da, non \(V\) sistema osoaren bolumena den.
\(\hat{b}_k \text{ eta } \hat{b}_k^\dagger\) hurrengo erlazioa beteko
dute

\begin{equation}\phantomsection\label{eq-p6.28}{
\hat{b}_k \hat{b}_k^\dagger - \hat{b}_k^\dagger \hat{b}_k = \delta_{kk'},
}\end{equation}

\(A_k\) koefizienteak aukeratu ditugu Hamiltondarra diagonala izateko.
\(\hat{\phi}\) eragilea Ekuazioa~\ref{eq-p6.24} erlazioa betetzeko

\begin{equation}\phantomsection\label{eq-p6.29}{
\hat{\phi} = -\frac{1}{\sqrt{2V}} \sum_{k \neq 0} i\hbar (A_k)^{-1} (\hat{b}_k e^{i\mathbf{k} \cdot \mathbf{r}} - \hat{b}_k^\dagger e^{-i\mathbf{k} \cdot \mathbf{r}}).
}\end{equation}

Ekuazioa~\ref{eq-p6.26} Hamiltondarra garatzen badugu, termino lineala
desagertu egingo da karkaren kontserbazioa dela eta. Orduan termino
koadratikoa:

\begin{equation}\phantomsection\label{eq-p6.30}{
H^{(2)} = \int \left( \frac{1}{2} \bar{\rho}(\nabla \hat{\phi})^2 + \frac{c^2}{2} \frac{\hat{\rho}^2}{\bar{\rho}} \right) d\mathbf{r},
}\end{equation}

non \(c^2 = (\rho/m) d\mu/d\rho\) soinuaren abiadura den.
Ekuazioa~\ref{eq-p6.27} eta Ekuazioa~\ref{eq-p6.29} aurreko
adierazpenean ordezkatuz, lortuko dugu gire Hamiltondarraren forma
diagonala

\begin{equation}\phantomsection\label{eq-p6.31}{
H^{(2)} = \sum_k \hbar \omega_k \hat{b}_k^\dagger \hat{b}_k,
}\end{equation}

izango dela. Ekuazioa~\ref{eq-p6.31} bat dator \(\omega_k = ck\)
dispertsio erlazioa duten fonoien Hamiltondarrarekin. Gainera, \(A_k\)
koefizienteen adierazpena lortu dezakegu

\begin{equation}\phantomsection\label{eq-p6.32}{
A_k = \left( \frac{\hbar k \bar{\rho}}{c} \right)^{1/2}.
}\end{equation}

\subsubsection{Superfluidoen biraketa eta bortize
kuantizatuak}\label{superfluidoen-biraketa-eta-bortize-kuantizatuak}

Superfluidoek, ezin dute solido zurrun gisa biratu. Biraketa bat lortuko
du bortize kuantizatuen bitartez. Nahiz eta, bortize baten
abiadura-eremua irrotazionala izan, bere zirkulazioa kuantizatuta dago:

\begin{equation}\phantomsection\label{eq-p6.89}{
\oint \mathbf{v}_s \cdot d\mathbf{l} = 2\pi \frac{\hbar}{m} s, \quad (s \in \mathbb{Z}),
}\end{equation}

non \(s\) bortizearen karga den. Hau behartzen du zirkulazioa \(h/m\)
unitateetan dagoela. Orduan bortez baten abiadura tangentziala:

\begin{equation}\phantomsection\label{eq-p6.90}{
v_s = s \frac{\hbar}{m} \frac{1}{r}
}\end{equation}

non \(r\) bortez-lerrotik dagoen distantzia den.

Bortize baten energia zinetikoa bere luzerarekiko, honela adieraz
daiteke:

\begin{equation}\phantomsection\label{eq-p6.92}{
E_v = L \pi \rho_s s^2 \left( \frac{\hbar}{m} \right)^2 \ln \left( \frac{R}{r_c} \right)
}\end{equation}

non \(R\) sistemaren da eta \(r_c\) bortizearen nukleoaren tamaina.
Bortizea energetikoki baimendua egoteko, gure sistema osoa \(\Omega\)
abiadura angeluar minimo batekin biratu beharko du:

\begin{equation}\phantomsection\label{eq-p6.93}{
\Omega_c = \frac{\hbar}{m R^2} \ln \left( \frac{R}{r_c} \right).
}\end{equation}

\(\Omega\) gero eta handiagoa eginez gero, gero eta bortize gehiagoren
zorrera ahalbidetzen da. Bortize kopuru maximoa \(\Omega\) jakin
batentzako:

\begin{equation}\phantomsection\label{eq-p6.95}{
n_v = \frac{m}{\pi \hbar} \Omega
}\end{equation}

izango da.

\texttt{\{\{\textless{}\ include\ SIMULACION/simulacion.qmd\ \textgreater{}\}\}}

\phantomsection\label{refs}
\begin{CSLReferences}{1}{0}
\bibitem[\citeproctext]{ref-Landau1941Superfluidity}
Landau, L. 1941. {«Theory of the Superfluidity of Helium II»}.
\emph{Phys. Rev.} 60 (abuztuak): 356--58.
\url{https://doi.org/10.1103/PhysRev.60.356}.

\bibitem[\citeproctext]{ref-Pitaevskii}
Pitaevskii, Lev, eta Sandro Stringari. 2016. \emph{Bose-Einstein
Condensation and Superfluidity}. Oxford University Press.
\url{https://doi.org/10.1093/acprof:oso/9780198758884.001.0001}.

\end{CSLReferences}




\end{document}
